\documentclass[10pt,a4paper, margin=1in]{article}
\usepackage{fullpage}
\usepackage{amsfonts, amsmath, pifont}
\usepackage{amsthm}
\usepackage{graphicx}

\usepackage[utf8]{inputenc}

\usepackage{float}
\usepackage{tkz-euclide}
\usepackage{tikz}
\usepackage{pgfplots}
\pgfplotsset{compat=1.13}

\usepackage{geometry}
 \geometry{
 a4paper,
 total={210mm,297mm},
 left=10mm,
 right=10mm,
 top=10mm,
 bottom=10mm,
 }
 \author{
  Yesilyurt, Yavuz Selim\\
  \texttt{e2259166@ceng.metu.edu.tr}
  \and
   Simsek, Halit\\
  \texttt{e2099760@ceng.metu.edu.tr}
}
\title{CENG 384 - Signals and Systems for Computer Engineers \\
Spring 2018-2019 \\
Written Assignment 3}

\begin{filecontents}{q1_a1.dat}
  k  ak
 -4  2
 -3  1
 -2  0
 -1  1  
  0  2
  1  1
  2  0
  3  1
\end{filecontents}

\begin{filecontents}{q1_a2.dat}
  k  ak
 -4  0
 -3  2
 -2  0
 -1  2
  0  0
  1  2
  2  0
  3  2
\end{filecontents}
 
\begin{filecontents}{q1_b1.dat}
  k  ak
 -4  3
 -3  2
 -2  1
 -1  2
  0  3
  1  2
  2  1
  3  2
\end{filecontents}
 
\begin{filecontents}{q1_b2.dat}
  k  ak
 -4  0
 -3  2
 -2  0
 -1  1  
  0  0
  1  2
  2  0
  3  1
\end{filecontents}

\begin{filecontents}{q2.dat}
  n  xn
 -4  2
 -3  2
 -2  0
 -1  0  
  0  2
  1  2
  2  0
  3  0
\end{filecontents}

\begin{document}
\maketitle

\noindent\rule{19cm}{1.2pt}

\begin{enumerate}

\item
	\begin{enumerate}
		\item
			To find Fourier series coefficients of $x[n]$ we will use following formula for 1 period:
			\begin{center}
			$a_k = \frac{1}{N}\sum_{n=<N>} x[n]e^{-jkw_0n}$ \\
			\end{center}
			And to be able to find them using this formula we need $N$ and $w_0=\frac{2\pi}{N}$. Checking the graph of $x[n]$ we see that $N=4$ and $w_0=\frac{\pi}{2}$. To find the Fourier Series coefficients let us use the period from 0 to 3:
			\begin{align*}
			a_k &= \frac{1}{4}\sum_{n=0}^{3} x[n]e^{-jk\frac{\pi}{2}n} \\
				&= \frac{1}{4}(x[0] + x[1]e^{-jk\frac{\pi}{2}} + x[2]e^{-2jk\frac{\pi}{2}} + x[3]e^{-3jk\frac{\pi}{2}}) \\
				&= 0 + \frac{1}{4}(cos(k\frac{\pi}{2}) - jsin(k\frac{\pi}{2})) + \frac{1}{2}(cos(k\pi) - jsin(k\pi)) + \frac{1}{4}(cos(3k\frac{\pi}{2}) - jsin(3k\frac{\pi}{2}))
			\end{align*}
			From above we can find coefficients:
			\begin{align*}
			a_0 &= \frac{1}{4} + \frac{1}{2} + \frac{1}{4} = 1 \\
			a_1 &= -\frac{j}{4} - \frac{1}{2} + \frac{j}{4} = -\frac{1}{2} \\
			a_2 &= -\frac{1}{4} + \frac{1}{2} -\frac{1}{4} = 0 \\
			a_3 &= \frac{j}{4} - \frac{1}{2} -\frac{j}{4} = -\frac{1}{2}			
			\end{align*}
			From periodicity we can simply say: $a_n = a_{n+4} = a_{n-4}$, so other coefficients can be found via this fact. \\\\
			So the magnitude spectrum of the coefficients: \\
\begin{figure}[H]
    \centering
    \ldots
    \begin{tikzpicture}[scale=1.0] 
      \begin{axis}[
          axis lines=middle,
          xlabel={$k$},
          ylabel={$|a_k|$},
          xtick={ -4, -3, ..., 3},
          ytick={0,1,2,3},
          yticklabels={0 , $\frac{1}{2}$, 1, $\frac{3}{2}$},
          ymin=0, ymax=3,
          xmin=-4, xmax=3,
          every axis x label/.style={at={(ticklabel* cs:1.05)}, anchor=west,},
          every axis y label/.style={at={(ticklabel* cs:1.05)}, anchor=south,},
          grid,
        ]
        \addplot [ycomb, black, thick, mark=*] table [x={k}, y={ak}] {q1_a1.dat};
      \end{axis}
    \end{tikzpicture}
    \ldots
    \caption{$k$ vs. $|a_k|$.}
    \label{fig:q1_a1}
\end{figure}
			So the phase spectrum of the coefficients: 
\begin{figure}[H]
    \centering
    \ldots
    \begin{tikzpicture}[scale=1.0] 
      \begin{axis}[
          axis lines=middle,
          xlabel={$k$},
          ylabel={$\angle a_k$},
          xtick={ -4, -3, ..., 3},
          ytick={-3, -2, ..., 3},
          yticklabels={$-\frac{3\pi}{2}$, $-\pi$, $-\frac{\pi}{2}$, 0, $\frac{\pi}{2}$, $\pi$, $\frac{3\pi}{2}$},
          ymin=-3, ymax=3,
          xmin=-4, xmax=3,
          every axis x label/.style={at={(ticklabel* cs:1.05)}, anchor=west,},
          every axis y label/.style={at={(ticklabel* cs:1.05)}, anchor=south,},
          grid,
        ]
        \addplot [ycomb, black, thick, mark=*] table [x={k}, y={ak}] {q1_a2.dat};
      \end{axis}
    \end{tikzpicture}
    \ldots
    \caption{$k$ vs. $\angle a_k$.}
    \label{fig:q1_a2}
\end{figure}
		\item
		\begin{enumerate}
			\item
				We need to add a negative impulse (i.e $-\delta(n)$) at every $n+1$ th point to ensure periodicity of $y[n]$. So we would end up something such as:
				\begin{center}
				$y[n] = x[n] - \sum_{k=-\infty}^{\infty} \delta(n+1 - 4k)$ \\
				\end{center}
			\item
				To find Fourier series coefficients of $y[n]$ we will use following formula for 1 period:
			\begin{center}
			$a_k = \frac{1}{N}\sum_{n=<N>} x[n]e^{-jkw_0n}$ \\
			\end{center}
			
			And to be able to find them using this formula we need $N$ and $w_0=\frac{2\pi}{N}$. Checking the graph of $y[n]$ we see that $N=4$ and $w_0=\frac{\pi}{2}$. To find the Fourier Series coefficients let us again use the period from 0 to 3:
			\begin{align*}
			a_k &= \frac{1}{4}\sum_{n=0}^{3} x[n]e^{-jk\frac{\pi}{2}n} \\
				&= \frac{1}{4}(x[0] + x[1]e^{-jk\frac{\pi}{2}} + x[2]e^{-2jk\frac{\pi}{2}} + x[3]) \\
				&= 0 + \frac{1}{4}(cos(k\frac{\pi}{2}) - jsin(k\frac{\pi}{2})) + \frac{1}{2}(cos(k\pi) - jsin(k\pi)) + 0
			\end{align*}
			From above we can find coefficients:
			\begin{align*}
			a_0 &= \frac{1}{4} + \frac{1}{2} = \frac{3}{4} \\
			a_1 &= -\frac{j}{4} - \frac{1}{2} = -\frac{1}{4}(j+2) \\
			a_2 &= -\frac{1}{4} + \frac{1}{2} = \frac{1}{4} \\
			a_3 &= \frac{j}{4} - \frac{1}{2} = \frac{1}{4}(j-2) 		
			\end{align*}
			From periodicity we can simply say: $a_n = a_{n+4} = a_{n-4}$, so other coefficients can be found via this fact. \\\\
			So the magnitude spectrum of the coefficients: 
\begin{figure}[H]
    \centering
    \ldots
    \begin{tikzpicture}[scale=1.0] 
      \begin{axis}[
          axis lines=middle,
          xlabel={$k$},
          ylabel={$|a_k|$},
          xtick={ -4, -3, ..., 3},
          ytick={0,1,2,3},
          yticklabels={0, $\frac{1}{4}$, $\sqrt{\frac{5}{16}}$, $\frac{3}{4}$},
          ymin=0, ymax=3,
          xmin=-4, xmax=3,
          every axis x label/.style={at={(ticklabel* cs:1.05)}, anchor=west,},
          every axis y label/.style={at={(ticklabel* cs:1.05)}, anchor=south,},
          grid,
        ]
        \addplot [ycomb, black, thick, mark=*] table [x={k}, y={ak}] {q1_b1.dat};
      \end{axis}
    \end{tikzpicture}
    \ldots
    \caption{$k$ vs. $|a_k|$.}
    \label{fig:q1_b1}
\end{figure}
			So the phase spectrum of the coefficients: 
\begin{figure}[H]
    \centering
    \ldots
    \begin{tikzpicture}[scale=1.0] 
      \begin{axis}[
          axis lines=middle,
          xlabel={$k$},
          ylabel={$\angle a_k$},
          xtick={ -4, -3, ..., 3},
          ytick={-2, -1, ..., 2},
		  yticklabels={$-1.14\pi$, $-0.85\pi$, 0, $0.85\pi$, $1.14\pi$},          
          ymin=-2, ymax=2,
          xmin=-4, xmax=3,
          every axis x label/.style={at={(ticklabel* cs:1.05)}, anchor=west,},
          every axis y label/.style={at={(ticklabel* cs:1.05)}, anchor=south,},
          grid,
        ]
        \addplot [ycomb, black, thick, mark=*] table [x={k}, y={ak}] {q1_b2.dat};
      \end{axis}
    \end{tikzpicture}
    \ldots
    \caption{$k$ vs. $\angle a_k$.}
    \label{fig:q1_b2}
\end{figure}
		\end{enumerate}
	\end{enumerate}	   
\item 
	First let us classify and order the given conditions for $x[n]$.
	\begin{enumerate}
		\item Says that $N=4$ and $w_0=\frac{\pi}{2}$
		\item Says that the sum of $x[n]$ for 2 periods is equal to 8. Then we would have:
			\begin{center}
			$x[0] + x[1] + x[2] + x[3] = 4$ \\
		 	\end{center}
		 \item Says that $a_{-3}=a^{*}_{15}$ which also means $a_1=a^{*}_3$, also it says that $|a_1-a_{11}|=1$ which also means $|a_1-a_3|=1$
		 \item Says that one of the coefficients is zero.
		 \item For this condition, if we would expand the $e$ terms inside the sum we would have the following:
		 	\begin{align*}
		 	e^{\frac{-j\pi k}{2}} &= cos(\frac{\pi k}{2}) -jsin(\frac{\pi k}{2}) \ \ (1)\\
		 	e^{\frac{-j\pi 3k}{2}} &= cos(\frac{\pi 3k}{2}) -jsin(\frac{\pi 3k}{2}) \ \ (2)
		 	\end{align*}
		 	Now subtract $2\pi$ from both $cos$ and $sin$ terms in equation (2) and add (1) and (2), result will be as follows:
		 	\begin{center}
		 	$2cos(\frac{\pi k}{2})$\\
		 	\end{center}
		 	Therefore this condition actually says $\sum_{k=0}^{3} x[k](2cos(\frac{\pi k}{2}))$, Furthermore calculation of this sum results in:
		 	\begin{center}
		 	$2x[0]-2x[2]=4$ namely $x[0]-x[2]=2$ \\
			\end{center}		 	 
	\end{enumerate}
	So first let us find Fourier series coefficients for $x[n]$ using $a_k = \frac{1}{N}\sum_{n=<N>} x[n]e^{-jkw_0n}$ formula:
	\begin{align*}
		a_k &= \frac{1}{4}(x[0]+x[1]e^{-jk\frac{\pi}{2}} + x[2]e^{-jk\pi} + x[3]e^{-3jk\frac{\pi}{2}}) \\ 
		a_0 &= \frac{1}{4}(x[0] + x[1] + x[2] + x[3]) \\ 
		a_1 &= \frac{1}{4}(x[0] + x[1](cos(\frac{\pi}{2}) - jsin(\frac{\pi}{2})) + x[2](cos(\pi) - jsin(\pi)) + x[3](cos(\frac{3\pi}{2}) - jsin(\frac{3\pi}{2})) \\ 
		a_2 &= \frac{1}{4}(x[0] + x[1](cos(\pi) - jsin(\pi)) + x[2](cos(2\pi) - jsin(2\pi)) + x[3](cos(3\pi) - jsin(3\pi)) \\ 
		a_3 &= \frac{1}{4}(x[0] + x[1](cos(\frac{3\pi}{2}) - jsin(\frac{3\pi}{2})) + x[2](cos(3\pi) - jsin(3\pi)) + x[3](cos(\frac{9\pi}{2}) - jsin(\frac{9\pi}{2}))
	\end{align*}
	Each of them yield as:
	\begin{align*}
		a_0 &= \frac{1}{4}(x[0] + x[1] + x[2] + x[3]) \\ 
		a_1 &= \frac{1}{4}(x[0] - jx[1] - x[2] + jx[3]) \\ 
		a_2 &= \frac{1}{4}(x[0] - x[1] + x[2] - x[3]) \\ 
		a_3 &= \frac{1}{4}(x[0] + jx[1] - x[2] - jx[3])
	\end{align*}
	From condition $a$ we see that $a_0 = \frac{1}{4}4 = 1$.\\ 
	From condition $c$ we know that if $a_1=a^{*}_3$ then they are conjugates of each other and they would be in a form such as:
		\begin{align*}
			a_1 &= a + bj \\
			a_3 &= a - bj
		\end{align*}
		and since $|a_1-a_3|=|2b|=1$ then $y\neq 0$ also $a_1\neq 0$ and $a_3\neq 0$. Since we have condition $d$ which says that one of the coefficients is zero, we deduce that $a_2=0$.\\
	For $a_1$ and $a_3$, let us subtract $a_3$ from $a_1$ and compare with condition $c$:
		\begin{align*}
			|a_1-a_3|&=1 \\
			\frac{1}{4}|2jx[3] - 2jx[1]| &= 1 \\
			|jx[3] - jx[1]| &= 2 \\
			x[3] - x[1] = 2 \ \text{or} \ x[1] - x[3] &= 2 \ \ \text{So we need to choose one of them}
		\end{align*}
		To be able to use in $a_3$ expansion let us choose $x[1] - x[3] = 2$, multiply this equation by $j$, $jx[1] - jx[3] = 2j$ and also, from condition $e$ we had $x[0]-x[2]=2$, add these up and we would end up with $a_3$ expansion, namely:
		\begin{center}
			We had $a_3 = \frac{1}{4}(x[0] + jx[1] - x[2] - jx[3])$ \\
			Now we know that $jx[1] - jx[3] = 2j$ and $x[0]-x[2]=2$ \\
			So $a_3 = \frac{1}{2} + \frac{1}{2}j$ \\
			and since $a_1$ was the conjugate of $a_3$ so $a_1 = \frac{1}{2} - \frac{1}{2}j$ \\
		\end{center}
		Now to determine the signal $x[n]$ we need to find its values for 1 period, using condition $b$, $e$ and value of $a_2$ we can find $x[0]$ and $x[2]$ as follows:
		\begin{align*}
		x[0] + x[1] + x[2] + x[3] &= 4 \ \ \text{from cond b}\\
		x[0] - x[1] + x[2] - x[3] &= 0 = a_2\\
		x[0] + x[2] &= 2 \\
		x[0] - x[2] &= 2 \ \ \text{from cond e} \\
		\text{So} \ \ x[0] = 2 \ \ &\text{and} \ \ x[2] = 0
		\end{align*}
		Using $a_0$ and $a_1$ we can also determine $x[1]$ and $x[3]$ as follows:
		\begin{align*}
		\frac{1}{4}(x[0] + x[1] + x[2] + x[3]) &= 1 = a_0\\
		2 + x[1] + 0 + x[3] &= 4\\
		x[1] + x[3] &= 2 \ \ (1)
		\end{align*}
		Also for $a_1$:
		\begin{align*}
		\frac{1}{4}(x[0] - jx[1] + x[2] + jx[3]) &= \frac{1}{2}-\frac{1}{2}j = a_1\\
		2 - jx[1] - 0 + jx[3] &= 2-2j\\
		j(-x[1] + x[3]) &= -2j\\
		x[3] - x[1] &= -2 \ \ (2)
		\end{align*}
		Equating (1) and (2) we get:
		\begin{align*}
		x[1] + x[3] &= 2 \ \ (1) \\
		x[3] - x[1] &= -2 \ \ (2) \\
		\text{So} \ \ x[1] = 2 \ \ &\text{and} \ \ x[3] = 0
		\end{align*}
		So in the and we have an $x[n]$ with $N=4$, $w_0=\frac{\pi}{2}$ and $x[0]=2$, $x[1]=2$, $x[2]=0$ and $x[3]=0$. Its graph is:
\begin{figure}[H]
    \centering
    \ldots
    \begin{tikzpicture}[scale=1.0] 
      \begin{axis}[
          axis lines=middle,
          xlabel={$n$},
          ylabel={$x[n]$},
          xtick={ -4, -3, ..., 3},
          ytick={-3, -2, ..., 3},
          ymin=-3, ymax=3,
          xmin=-4, xmax=3,
          every axis x label/.style={at={(ticklabel* cs:1.05)}, anchor=west,},
          every axis y label/.style={at={(ticklabel* cs:1.05)}, anchor=south,},
          grid,
        ]
        \addplot [ycomb, black, thick, mark=*] table [x={n}, y={xn}] {q2.dat};
      \end{axis}
    \end{tikzpicture}
    \ldots
    \caption{$n$ vs. $x[n]$.}
    \label{fig:q2}
\end{figure}

\item 
	We are needed to find $h(t)$ where $x(t) = h(t)*(x(t)+r(t))$. If we would take the Fourier Transform of this equation, we will end up with:
	\begin{align*}
	X(jw) &= H(jw) \times (X(jw)+R(jw)) \\	
	X(jw) &= H(jw) \times X(jw) + H(jw) \times R(jw) \\
	X(jw) &= H(jw) \times X(jw) \ \ \text{(since} \ R(jw)=0\text{)}\\ 
	H(jw) &= 1
	\end{align*}
	Since we know $H(jw)$, we will now use $h(t) = \frac{1}{2\pi}\int_{-\infty}^{\infty} H(jw)e^{jwt}dw$ to find $h(t)$:
	\begin{align*}
	h(t) &= \frac{1}{2\pi}\int_{-\infty}^{\infty} H(jw)e^{jwt}dw \\
 		 &= \frac{1}{2\pi}\int_{-\frac{K2\pi}{T}}^{\frac{K2\pi}{T}} 1e^{jwt}dw \\
		 &= \frac{1}{2\pi}|^{\frac{K2\pi}{T}}_{-\frac{K2\pi}{T}}\frac{e^{jwt}}{jt} \\
		 &= \frac{1}{2\pi}(\frac{e^{\frac{jK2\pi t}{T}}}{jt} - \frac{e^{-\frac{jK2\pi t}{T}}}{jt}) \\
		 &= \frac{1}{2\pi jt}(cos(\frac{K2\pi t}{T})+jsin(\frac{K2\pi t}{T}) - cos(\frac{K2\pi t}{T})+jsin(\frac{K2\pi t}{T})) \\
		 &= \frac{sin(\frac{K2\pi t}{T})}{\pi t} 
	\end{align*}
\item 
	\begin{enumerate}
		\item
			First let us find the system equation from block diagram:
	\begin{align*}
		y^{'}(t)&=-5y(t)+4x(t) + \int x(t)-6y(t) dt\\
		\text{namely:} \ \ y^{''}(t)&=-5y^{'}(t)+4x^{'}(t) + x(t)-6y(t) 
	\end{align*}
	We will find frequency response $H(jw)$ by giving input $x(t) = e^{jwt}$, so we would have $y(t)=H(jw)\times e^{jwt}$ as the output. So plugging in the values to the system:
	\begin{align*}
	(jw)^2 e^{jwt} H(jw) &= -5H(jw)jw e^{jwt} + 4jw e^{jwt} + e^{jwt} - 6H(jw)e^{jwt} \\
	H(jw)e^{jwt}(j^2w^2 + 5jw + 6) &= 4jwe^{jwt} + e^{jwt} \\
	H(jw) &= \frac{4jw+1}{j^2w^2+5jw+6} 
	\end{align*}
	\item
		We know that $H(jw) \xrightarrow{\mathfrak{F}^{-1}} h(t)$, so all we need to do is to take inverse Fourier transform of $H(jw)$. It can be done as follows:
		\begin{align*}
		H(jw) &= \frac{4jw+1}{j^2w^2+5jw+6} \\
		      &= \frac{4jw+1}{(jw+3)(jw+2)} \\
		\frac{4jw+1}{(jw+3)(jw+2)} &= \frac{A}{jw+3} + \frac{B}{jw+2} \\ 
		4jw + 1 &= Ajw + 2A + Bjw + 3B 
		\end{align*}
		Equating both sides according to their Re and Im parts:
		\begin{align*}
		A + B &= 4 \\
		2A + 3B &= 1 \\
		\text{we get:} \ B=-7 &\text{ and } A=11
		\end{align*}
		\begin{center}
		So $H(jw)=\frac{11}{jw+3} - \frac{7}{jw+2}$	\\
		\end{center}
		
		Now take the $\mathfrak{F}^{-1}$ of $H(jw)$ (using $\mathfrak{F}$ table):
		\begin{center}
			$H(jw)=\frac{11}{3+jw} - \frac{7}{2+jw} \xrightarrow{\mathfrak{F}^{-1}} h(t)=11e^{-3t}u(t)-7e^{-2t}u(t)$ \\ 
		\end{center}		
	\item
		To be able to find $y(t)$ for $x(t)=\frac{1}{4}e^{-\frac{t}{4}}u(t)$ using $H(jw)$, we need to first take $\mathfrak{F}$ of $x(t)$ and get $X(jw)$, then using multiplication (instead of convolution) find $Y(jw)$ and then again take $\mathfrak{F}^{-1}$ of $Y(jw)$ and find $y(t)$. This can be done as follows:
		\begin{align*}
		\text{From} \ \mathfrak{F} \ \text{table we know that:} \ e^{-|a|t}u(t) &\xrightarrow{\mathfrak{F}} \frac{1}{|a|+jw} \\		
		\text{So} \ \ \frac{1}{4}e^{-\frac{t}{4}}u(t) &\xrightarrow{\mathfrak{F}} \frac{1}{4}\times \frac{1}{\frac{1}{4}+jw} = X(jw) \\
		\end{align*}
		Now find $Y(jw)$ using multiplication with $H(jw)$:
		\begin{align*}
		 Y(jw) &= H(jw)\times X(jw) \\
		      &= \frac{4jw+1}{(jw+3)(jw+2)} \times \frac{1}{1+4jw}\\
		      &= \frac{1}{(jw+3)(jw+2)} 
		\end{align*}
		Now take the $\mathfrak{F}^{-1}$ of $Y(jw)$ (again using $\mathfrak{F}$ table):
		\begin{align*}
		Y(jw) &= \frac{1}{jw+3} - \frac{1}{jw+2}\\
		\frac{1}{3+jw} - \frac{1}{2+jw} &\xrightarrow{\mathfrak{F}^{-1}} y(t)=e^{-3t}u(t)-e^{-2t}u(t) = (e^{-2t}-e^{-3t})u(t)
		\end{align*}
	\end{enumerate}
\end{enumerate}
\end{document}
