\documentclass[10pt,a4paper, margin=1in]{article}
\usepackage{fullpage}
\usepackage{amsfonts, amsmath, pifont}
\usepackage{amsthm}
\usepackage{graphicx}

\usepackage[utf8]{inputenc}

\usepackage{float}
\usepackage{tkz-euclide}
\usepackage{tikz}
\usepackage{pgfplots}
\pgfplotsset{compat=1.13}

\usepackage{geometry}
 \geometry{
 a4paper,
 total={210mm,297mm},
 left=10mm,
 right=10mm,
 top=10mm,
 bottom=10mm,
 }
 \author{
  Yesilyurt, Yavuz Selim\\
  \texttt{e2259166@ceng.metu.edu.tr}
  \and
   Simsek, Halit\\
  \texttt{e2099760@ceng.metu.edu.tr}
}
\title{CENG 384 - Signals and Systems for Computer Engineers \\
Spring 2018-2019 \\
Written Assignment 2}


\begin{document}
\maketitle

\noindent\rule{19cm}{1.2pt}

\begin{enumerate}

\item 
    \begin{enumerate}
    \item
    $y^{'}(t)=x(t) - 4y(t)$ \\
    \item
    We have input $x(t)=(e^{-t}+e^{-2t})u(t)$ and we are given the information that the system is initially at rest. \\
    
    We can find output $y(t)$ with computing and adding homogoneous and particular solutions, i.e. adding the outputs of the system when there is no input ($y_h(t)$) and when a particular input $x(t)$ is given to the system ($y_p(t)$). So: 
	\begin{center}
	$y(t)=y_h(t)+y_p(t)$ 
	\end{center}
	
	To find $y_h(t)$ we will hypothesize a solution of the form of an exponential, such that: $y_h(t) = Ke^{\alpha}t$. So solving the equation for $y_h(t)$: 
	\begin{align*}
	y^{'}+4y(t)&=0 \\
	(\alpha+4)Ke^{\alpha}t&=0 \\
	\alpha + 4 &= 0 \\
	\alpha &= -4  \\
	y_h(t)&=Ce^{-4t} \ \ \ \ \ \text{Substituting} \ \ \alpha \ \ \text{we get} \ \ y_h(t)
	\end{align*}
	
	For particular solution, we will hypothesize a solution of the form of an exponential, such that: $y_p(t) = Ae^{-t}+Be^{-2t}$. So solving the equation for $y_p(t)$:
	
	\begin{align*}
	y^{'}+4y(t)&=x(t) \\
	-Ae^{-t}-2Be^{-2t}+4Ae^{-t}+4Be^{-2t}&=e^{-t}+e^{-2t}\\
	3Ae^{-t}+2Be^{-2t} &= e^{-t}+e^{-2t} \\
	3A = 1 \ \ &\text{and} \ \ 2B = 1 \\
	A = \frac{1}{3} \ \ &\text{and} \ \ B = \frac{1}{2}\\
	y_p(t)&=\frac{1}{3}e^{-4t} + \frac{1}{2}e^{-2t} \ \ \ \ \ \text{Substituting} \ A \ \text{and} \ B \ \text{we get} \ \ y_p(t)
	\end{align*}
	
	So we have:
	\begin{center}
	$y(t)=Ce^{-4t}+\frac{1}{3}e^{-4t} + \frac{1}{2}e^{-2t}$ 
	\end{center}
	
	Since system is initially at rest, we have initial conditions such as: $y(0)=y^{'}(0)=...=0$, Solving for $y(t)$:
	\begin{align*}
	y(0) &= C + \frac{1}{3} + \frac{1}{2} = 0\\
	C &= -\frac{5}{6}
	\end{align*}
    
    So the answer:
    
    \begin{center}
	$y(t)=-\frac{5}{6}e^{-4t}+\frac{1}{3}e^{-t} + \frac{1}{2}e^{-2t}$ 
    \end{center}
    \end{enumerate}
\newpage
\item 
    \begin{enumerate}
    \item
    Use distributive property of Convolution:
    \begin{align*}
    y(n) &= x(n) * h(n) \\
         &= \delta(n-1)(h(n)) - 3\delta(n-2)(h(n)) + \delta(n-3)(h(n)) \\
         &= \delta(n+1) - \delta(n) - 8\delta(n-1) + 11\delta(n-2) - 3 \delta(n-3)
    \end{align*}
    \item
    We have $y(t)=x^{'}(t)*h(t)$ and $x(t)=u(t)+u(t-1)$, we can easily find the derivative of $x(t)$ since it is composed of unit step functions. We will have: 
    \begin{align*}
   	 x(t)&=u(t)+u(t-1) \\
	 x^{'}(t)&=\delta(t) + \delta(t-1) 
    \end{align*}
    To calculate $y(t)$, we can again use the distributive property of Convolution:
    \begin{align*}
    y(t)&=x^{'}(t)*h(t) \\
         &= \delta(t)(h(t)) + \delta(t-1)(h(t)) \\
         &= h(t) + h(t-1)\\
         &= (e^{-2t}cost+e^{-2t+2}cos(t-1))u(t)
    \end{align*}
    \end{enumerate}
    
\item 
    \begin{enumerate}
    \item
    We need to calculate $y(t)=x(t)*h(t)=\int_{\infty}^{-\infty} x(\tau)h(t-\tau)d\tau$. \\
    Plugging the values in:
    \begin{align*}
    y(t)&=x(t)*h(t) \\
    	&=\int_{\infty}^{-\infty} x(\tau)h(t-\tau)d\tau \\
    	&=\int_{t}^{0} e^{-\tau}e^{-3t+3\tau}d\tau \\
    	&=e^{-3t}\int_{t}^{0} e^{2\tau}d\tau \\
    	&=\frac{e{-t}-e{-3t}}{2}u(t)
    \end{align*}
    \item
    We are given $h(t) = e^tu(t)$ and $x(t)=u(t-1)-u(t-2)$. For calculation of $y(t)=x(t)*h(t)$ we need to divide $x(t)$ into three different regions since $x(t)$ actually behaves differently for $t>2$, $1\leq t \leq 2$ and $t<1$. \\
    
    For $t<1$, we can easily conclude that $y(t)=0$ since $x(t)$ and $h(t)$ has no intersecting area. \\
    
    For $1 \leq t \leq 2$ part we have $x(t) = 1$ : 
    \begin{align*}
        y(t) &= x(t)*h(t) \\
        &= \int_1^t 1 \times e^{t - \tau}d\tau \\
        &= e^t \int_1^t e^{-\tau}d\tau \\
        &= |_1^t-e^t \times e^{-t} \\
        &= -1 + e^{t - 1} 
    \end{align*}
    For $2 < t$ part we have again $x(t) = 1$ : 
    \begin{align*}
        y(t) &= x(t)*h(t) \\
        &= \int_1^2 1 \times e^{t - \tau}d\tau \\
        &= e^t \int_1^2 e^{-\tau}d\tau \\
        &= |_1^2-e^t \times e^{-\tau}\\
        &= -e^{t - 2} + e^{t - 1}
    \end{align*}
	Combining all three cases, we have the following as the result of Convolution $y(t)$:
    \begin{align*}
        y(t) &= 0\ ,\ t < 1 \\
        y(t) &= -1 + e^{t - 1}\ ,\ 1 \leq t \leq 2 \\
        y(t) &= -e^{t - 2} + e^{t - 1}\ ,\ 2 < t \\
    \end{align*}
    \end{enumerate}
\newpage

\item 
    \begin{enumerate}
    \item
   	Getting the characteristic equation of $y(n)$ we get:
   	\begin{center}
   	$P(\lambda)=\lambda^2-15\lambda+26=0$, getting the roots of the equation we get: \\
   	$\lambda_1=2$ and $\lambda_2=13$, so we have a general solution such as: \\
   	$y(n)=A2^{n}+B13^{n}$
   	\end{center}
   	To find $A$ and $B$ we will use the initial conditions: 
   	\begin{center}
   	$y(0)=A+B=10$ and $y(1)=2A+13B=42$, from here we get values as the following: \\
   	$A=8$ and $B=2$
   	\end{center}
   	So the general solution results as the following: \\
   	\begin{center}
   	$y(n)=8\times 2^{n}+2\times 13^{n}$
   	\end{center}
    \item
    Getting the characteristic equation of $y(n)$ we get:
   	\begin{center}
   	$P(\lambda)=\lambda^2-3\lambda+1=0$, getting the roots of the equation we get: \\
   	$\lambda_1=(\frac{3+\sqrt{5}}{2})$ and $\lambda_2=(\frac{3-\sqrt{5}}{2})$, so we have a general solution such as: \\
   	$y(n)=A(\frac{3+\sqrt{5}}{2})^{n}+B(\frac{3-\sqrt{5}}{2})^{n}$
   	\end{center}
   	To find $A$ and $B$ we will use the initial conditions: 
   	\begin{center}
   	$y(0)=A+B=1$ and $y(1)=A(\frac{3+\sqrt{5}}{2})+B(\frac{3-\sqrt{5}}{2})=2$, from here we get values as the following: \\
   	$A=\frac{\sqrt{5}+1}{2\sqrt{5}}$ and $B=\frac{\sqrt{5}-1}{2\sqrt{5}}$
   	\end{center}
   	So the general solution results as the following: \\
   	\begin{center}
   	$y(n)=\frac{\sqrt{5}+1}{2\sqrt{5}} \times (\frac{3+\sqrt{5}}{2})^{n}+ \frac{\sqrt{5}-1}{2\sqrt{5}} \times (\frac{3-\sqrt{5}}{2})^{n}$
   	\end{center}
    \end{enumerate}
    
\item 
    \begin{enumerate}
    \item
    To find the impulse response $h(t)$ for the system, let's first find impulse response $w(t)$ for a relatively easy system such as:
    \begin{center}
    $y^{''}(t)+6y{'}(t)+8y(t)=x(t)$
    \end{center}
    No particular solution for $t>0$ since, for $x(t)=\delta(t)=0$ for $t > 0$. Let us find the response $w(t)$, assuming a general form as $w(t) = Ae^{st}$ for $t > 0$: 
    \begin{align*}
     w^{''}(t)+6w{'}(t)+8w(t)&=0 \\
     As^2e^{st}+6Ase^{st}+8Ae^{st}&=0 \\
     Ae^{st}(s^2+6s+8)&=0 \\
     s_1 = -4 \ \ \text{and} \ \ s_2 &= -2
    \end{align*}
    So we get:
    \begin{center}
    $w(t)=Ae^{-4t}+Be^{-2t}$ for $t > 0$
    \end{center}
    To obtain the coefficients, we will use the initial conditions, since the system is initially at rest we have:
    \begin{center}
    $y_h^{n-1}(0)=1$ and $y_h^{n-2}(0)=y_h^{n-3}(0)=...=y_h(0)=0$
    \end{center}
  	So plugging the values in to get the coefficients:
	\begin{align*}
	w^{'}(0)&=-4A-2B=1 \\
	w(0)&=A+B=0 \\
	-2A &= 1 \\
	A=-\frac{1}{2} \ \ &\text{and} \ \ B=\frac{1}{2}	
	\end{align*}
	So we have the impulse response $w(t)$ for the easy system as the following:
	\begin{center}
	$w(t)=-\frac{1}{2}e^{-4t}+\frac{1}{2}e^{-2t}$
	\end{center}
	If this system has impulse response $w(t)$ then the original system must have $h(t)=2w(t)$:
	\begin{center}
	$h(t)=(-e^{-4t}+e^{-2t})u(t)$ \\
	\end{center}
    \item
    	\begin{enumerate}
    	\item
        LTI is causal if $h(t)=0$ for $t<0$, since our $h(t)=(-e^{-4t}+e^{-2t})u(t)$ is multiplied by unit step function $u(t)$, it is equal to 0 for $t<0$, so $h(t)$ is causal. \\
        \item
        LTI is memoryless if $h(t)=K\delta(t)$, since our $h(t)=(-e^{-4t}+e^{-2t})u(t)$ can't be written in such a form and in addition, since our $h(t)$ needs memory for (not for $t<0$ since it equals to 0 in there) $t>0$, it is not memoryless, instead it needs memory. \\
        \item
        LTI is stable if $h(t)$ is finitely integrable, i.e:
        \begin{center}
        $\int_{\infty}^{-\infty}|h(\tau)|d\tau < \beta$
        \end{center}
        We have:
        \begin{align*}
        \int_{\infty}^{-\infty}(-e^{-4\tau}+e^{-2\tau})u(\tau)d\tau &< \beta \\
        \int_{\infty}^{0}-e^{-4\tau}+e^{-2\tau}d\tau &< \beta \\
        |_0^{\infty}-\frac{e^{-4\tau}}{4} - \frac{e^{-2\tau}}{2} &< \beta \\
        \frac{1}{4} &< \beta
        \end{align*}
        Since $\frac{1}{4}$ is a constant value, we can say $h(t)$ is bounded, finitely integrable and stable. \\
        \item
        LTI is invertible when $h(t)*h^{-1}(t)=\delta(t)$ and when, for $y(t)=x(t)*h(t)$ the inverse of the $h(t)$ results in $x(t)=y(t)*h^{-1}(t)$. We have $h(t)=(-e^{-4t}+e^{-2t})u(t)$, when we examine $h(t)$ by checking its graph, we see that it gives same output for more than one distinct inputs, Moreover, we also see that it has points where the derivative of the function is zero. Hence $h(t)$ can not be/is not invertible.
    	\end{enumerate}
    \end{enumerate}

\end{enumerate}
\end{document}
