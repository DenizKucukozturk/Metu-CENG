\documentclass[10pt]{article}


%\topmargin=0cm
\topmargin=-2cm
\addtolength{\textheight}{6.5cm}
\addtolength{\textwidth}{2.0cm}

\begin{document}

\section*{Student Information } 
Full Name : Yavuz Selim Yesilyurt \\
Id Number : 2259166 \\

\section*{Answer 1}
\hspace{4mm}	
IP address = 144.122.145.146 and MAC address = 00:00:c0:f0:4e:f9

\section*{Answer 2}
\hspace{4mm}	
Number = 45, Time = 8.656587229, Destination address = 144.122.145.146

\section*{Answer 3}
\hspace{4mm}	
Number of 1st HTTP request = 45, Time of 1st HTTP request = 8.656587229

Number of 2nd HTTP request = 85, Time of 2nd HTTP request = 8.974345863

Number of 3rd HTTP request = 86, Time of 3rd HTTP request = 8.974396851

Number of 4th HTTP request = 87, Time of 4th HTTP request = 8.974426990

Number of 5th HTTP request = 88, Time of 5th HTTP request = 8.974453629


\section*{Answer 4}
\hspace{4mm}	
   Number of 1st HTTP response = 64, Time of 1st HTTP response = 8.908823256 (mathing response of first HTTP request)
   
   Number of 2nd HTTP response = 96, Time of 2nd HTTP response = 8.979147442 (mathing response of second HTTP request)

   Number of 3rd HTTP response = 105, Time of 3rd HTTP response = 8.980336274 (mathing response of third HTTP request)

   Number of 4th HTTP response = 111, Time of 4th HTTP response = 8.987639900 (mathing response of fourth HTTP request)

   Number of 5th HTTP response = 126, Time of 5th HTTP response = 8.987703730 (mathing response of fifth HTTP request)

\section*{Answer 5}
\hspace{4mm}	
We can match a HTTP request and a HTTP response on Wireshark environment with the following way:
   We can filter captured packages according to their destination IP addresses (my own IP address), source IP addresses (server IP address), Protocol used (HTTP) and their TCP ports.
   To make such a filtering, I first get the TCP port numbers of corresponding HTTP requests from their TCP headers
   and then I included them on my each filtering with "tcp.port" keyword. The filterings that I have performed are the following:\\
   
   "ip.dst == 10.70.196.158 and ip.src == 144.122.145.146 and http and tcp.port == 47672"
   
   "ip.dst == 10.70.196.158 and ip.src == 144.122.145.146 and http and tcp.port == 47674"

   "ip.dst == 10.70.196.158 and ip.src == 144.122.145.146 and http and tcp.port == 47676"

   "ip.dst == 10.70.196.158 and ip.src == 144.122.145.146 and http and tcp.port == 47678"

   "ip.dst == 10.70.196.158 and ip.src == 144.122.145.146 and http and tcp.port == 47680"


\section*{Answer 6}
\hspace{4mm}	
Web browser uses a non-persistent HTTP connection since it opens a new TCP connection for each request-response operation. After the 1 request-response operation is done, the connection gets dropped. Then, for a new request-response operation a new TCP connection gets created with a new TCP Port and used.  

\end{document}

​

