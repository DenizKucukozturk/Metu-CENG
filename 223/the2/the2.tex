\documentclass[12pt]{article}
\usepackage[utf8]{inputenc}
\usepackage{float}
\usepackage{amsmath}


\usepackage[hmargin=3cm,vmargin=6.0cm]{geometry}
%\topmargin=0cm
\topmargin=-2cm
\addtolength{\textheight}{6.5cm}
\addtolength{\textwidth}{2.0cm}
%\setlength{\leftmargin}{-5cm}
\setlength{\oddsidemargin}{0.0cm}
\setlength{\evensidemargin}{0.0cm}

%misc libraries goes here
\usepackage{amssymb}

\begin{document}

\section*{Student Information } 
%Write your full name and id number between the colon and newline
%Put one empty space character after colon and before newline
Full Name : Yavuz Selim YEŞİLYURT \\
Id Number : 2259166 \\

% Write your answers below the section tags
\section*{Answer 1}
\hspace{5mm} a) Using membership table to prove $A \cap B \subseteq (A \cup \overline B) \cap (\overline A \cup B)$:
\begin{table}[H]
\small
\centering
\begin{tabular}{|c|c|c|c|c|c|c|c|c}	
\hline
$A$ & $B$ & $ \overline A $ & $ \overline B$ & $A \cap B$ & $A \cup \overline B$ & $\overline A \cup B$ & $ (A \cup \overline B) \cap (\overline A \cup B)$\\
\hline 
1 & 1 & 0 & 0 & 1 & 1 & 1 & 1\\			
1 & 0 & 0 & 1 & 0 & 1 & 0 & 0\\
0 & 1 & 1 & 0 & 0 & 0 & 1 & 0\\
0 & 0 & 1 & 1 & 0 & 1 & 1 & 1\\
\hline 
\end{tabular}
\end{table}
b) Using membership table to prove $\overline A \cap \overline B \subseteq (A \cup \overline B) \cap (\overline A \cup B)$:
\begin{table}[H]
\small
\centering
\begin{tabular}{|c|c|c|c|c|c|c|c|c}	
\hline
$A$ & $B$ & $ \overline A $ & $ \overline B$ & $\overline A \cap \overline B $ & $A \cup \overline B$ & $\overline A \cup B$ & $ (A \cup \overline B) \cap (\overline A \cup B)$\\
\hline 
1 & 1 & 0 & 0 & 0 & 1 & 1 & 1\\			
1 & 0 & 0 & 1 & 0 & 1 & 0 & 0\\
0 & 1 & 1 & 0 & 0 & 0 & 1 & 0\\
0 & 0 & 1 & 1 & 1 & 1 & 1 & 1\\
\hline 
\end{tabular}
\end{table}

\section*{Answer 2}
\begin{align*}
f^{-1}((A \cap B) \times C) &= f^{-1}(A \times B) \cap f^{-1}(B \times C) \\
f^{-1}((A \times C) \cap (B \times C)) &= f^{-1}(A \times C) \cap f^{-1}(B \times C) \\
\end{align*} 

\hspace*{5mm}\textbf{Part 1}, show that $f^{-1}((A \times C) \cap (B \times C)) \subseteq  f^{-1}(A \times C) \cap f^{-1}(B \times C)$;\\
Assume that $x \in f^{-1}((A \times C) \cap (B \times C))$, then $f(x) \in ((A \times C) \cap (B \times C))$,\\ so $f(x) \in (A \times C)$ and $f(x) \in (B \times C)$, therefore $x \in f^{-1}(A \times C)$ and $x \in  f^{-1}(B \times C)$,\\ so $ x \in ( f^{-1}(A \times C) \cap  f^{-1}(B \times C))$\\\\

\hspace*{5mm}\textbf{Part 2}, show that $ f^{-1}(A \times C) \cap f^{-1}(B \times C) \subseteq f^{-1}((A \times C) \cap (B \times C))$;\\
Assume that $x \in f^{-1}(A \times C) \cap f^{-1}(B \times C)$, then $x \in  f^{-1}(A \times C)$ and $x \in  f^{-1}(B \times C)$,\\ so $f(x) \in (A \times C)$ and $f(x) \in (B \times C)$, which implies $ f(x) \in ((A \times C) \cap (B \times C))$, \\ therefore $x \in  f^{-1}((A \times C) \cap (B \times C))$\\\\

\hspace*{5mm}Hence by part1 and part2 $f^{-1}((A \times C) \cap (B \times C)) =  f^{-1}(A \times C) \cap f^{-1}(B \times C)$\\
\hspace*{5mm}\quad \ Therefore $f^{-1}((A \cap B) \times C) = f^{-1}(A \times B) \cap f^{-1}(B \times C)$

\section*{Answer 3}
\hspace*{5mm}a)	 From the definition of one to one functions, if a function is one to one then for all x and for all y; if $f(x) = f(y)$ then $x = y$. We can clearly see this is not the case for this function. Since the function has $x^2$ term in its interior, it maps the negative items from $R$ to their positive equivalents in $R$ (i.e $(-x)^2 = x^2$ while $x \neq -x$ , even function). So we see that there is at least one situation which contradicts with our assumption, therefore $f(x)$ can not be one-to-one. We know that $f : A \rightarrow B$ is onto if $f(A) = B$. Which means if a function f is onto then every element in its codomain is being mapped. This function is defined from $R$ to $R$ so if $f(x)$ is onto then every item on $R$ must be mapped by $f(x)$. But negative values on $R$ will not be mapped by $f(x)$ . Therefore $f(x)$ can not be onto.\\\\
\hspace*{5mm}b)	From the definitions of one-to-one functions and onto functions on part a, $f(x)$ is one-to-one, since the function maps every distinct item from $R$ to a distinct item in $R$. But $f(x)$ is not onto, since the $0$ in $R$ will remain unmapped from $f(x)$. Therefore $f(x)$ can not be onto.


\section*{Answer 4}

\hspace{6mm}a)	
Let A and/or B countably infinite sets such that $A=\{a_1,a_2...\}$ and
$B=\{b_1,b_2...\}$ we can write the elements of $A\times B$ as;\\\\
\begin{tabular}{ c| c c c c c}
 $A \setminus B$  &$ b_1$ & $b_2$ & $b_3$ & ...  \\
   \hline
 $ a_1$ & $(a_1,b_1)$ & $(a_1,b_2)$ & $(a_1,b_3)$ &...\\
 $ a_2$ & $(a_2,b_1)$ & $(a_2,b_2)$ & $(a_2,b_3)$ &...\\
 $ a_3$ & $(a_3,b_1)$ & $(a_3,b_2)$ & $(a_3,b_3)$ &...\\
 ... & ... & ... & ... & ...
\end{tabular}
\\\\
In table, first row, above the horizontal line contains the countably infinite elements of set $B$, first column, left of vertical line contains the countably infinite elements of set $A$. We can write $A \times B$ on this table and we can count the elements of $A \times B$ diagonally (i.e. 1st element: $(a_1,b_1)$, 2nd element: $(a_2,b_1)$, 3rd element $(a_1,b_2)$, 4th element: $(a_1,b_3)$, 5th element: $(a_2,b_2)$, 6th element: $(a_3,b_1)$ and so on. ). With this way of counting we will eventually get all elements in $A \times B$, then we can index these elements with unique numbers from $\mathbb{N}$, therefore we can get the one-to-one correspondence between $A \times B$ and  $\mathbb{N}$. Therefore we can infer that $A \times B$ is countable.\\

b)	Prove by contradiction, suppose that $B$ were countable, say with elements $b1, b2 , ...$ .Then since $A \subseteq B$ , we can list the elements of $A$ using the order in which they appear in this listing of $B$. Therefore $A$ is countable. But this contradicts with the hypothesis ($A$ is uncountable). Thus $B$ is not countable.\\

c)	If $B$ is countable with the elements $b1, b2 , ...$ and $A \subseteq B$, then we can list the elements of $A$ using the order in which they appear in this listing of $B$. Therefore $A$ is countable.

\newpage

\section*{Answer 5}
\hspace{6mm}a)	Let $f_1(x) = 7x^2$ and $f_2(x)=x^3$ ($f_1(x)$ is $\mathcal{O}(f_2(x)$) and they are increasing functions). \\ 
Prove by the Big-Oh definition,  $ln|f_1(x)|$ is $\mathcal{O}(ln|f_2(x)|)$ if $ln|f_1(x)|\leq  c.(ln|f_2(x)|)$ for some $x>k$. Let us check this condition: If  $ln|7x^2|$ is $\mathcal{O}(ln|x^3|)$ then $ln|7x^2|\leq c.ln|x^3|$ . Exponential both sides  $e^{ln|7x^2|} \leq e^{c.ln|x^3|}$, eliminate the terms  $7x^2 \leq e^c.x^3$, note that when $x\geq k=7$ and $c \geq 1$, we have $7x^2 \leq e^c.x^3$, consequently, we can take $c=1$ and $k=7$ as witnesses to establish the relationship $ln|7x^2|$ is $\mathcal{O}(ln|x^3|)$. Therefore $ln|f_1(x)|$ is $\mathcal{O}(ln|f_2(x)|)$.\\

b)	Let $f_1(x) = 7x^2$ and $f_2(x)=x^3$ ($f_1(x)$ is $\mathcal{O}(f_2(x)$) and they are increasing functions). \\
Prove by the Big-Oh definition, $3^{f_1(x)}$ is $\mathcal{O}(3^{f_2(x)})$ if  $3^{f_1(x)} \leq c.3^{f_2(x)}$ for some $x>k$. Let us check this condition: If $3^{7x^2}$ is $\mathcal{O}(3^{x^3})$ then $3^{7x^2} \leq c.3^{x^3}$ . Take the log of both sides $7x^2.log3 \leq x^3.log3+logc$, divide log3 both sides $7x^2 \leq x^3 + log(c-3)$, note that when $x\geq k=7$ and $c \geq 4$, we have $7x^2 \leq x^3 + log(c-3)$, consequently, we can take $c=4$ and $k=7$ as witnesses to establish the relationship $3^{7x^2}$ is $\mathcal{O}(3^{x^3})$. Therefore $3^{f_1(x)}$ is $\mathcal{O}(3^{f_2(x)})$.\\

\section*{Answer 6}
\hspace{6mm}a)
\begin{equation*}
\begin{split}
(3^x -1 )\ mod(3^y -1) &= 3^{(x \ mod \ y)} -1 \\
3^x -1		 &= k.(3^y -1) + 3^{(x \ mod \ y)}-1 \ \ \ \ \ \ \ where \  k \in \mathbb{Z} \\
3^x 			 &= k.(3^y -1) + 3^{(x \ mod \ y)} \\
log_3(3^x)		 &= log_3(k.(3^y -1) + 3^{(x \ mod \ y)}) \\
x			 &= log_3(k.(3^y -1) + 3^{(x \ mod \ y)})  \\
\end{split}	 
\end{equation*} \\ 
\quad Now substitute $x$ in original equation:\\
\begin{equation*}
\begin{split}
(3^{log_3(k.(3^y -1) + 3^{(x \ mod \ y)})}-1)\ mod(3^y -1) &= 3^{(x \ mod \ y)} -1 \\
k.(3^y -1) + 3^{(x \ mod \ y)}-1		 				&= k.(3^y -1) + 3^{(x \ mod \ y)}-1  \\
0											&= 0 \ \ \ \ \ \ \ (1) \\
\end{split}	 
\end{equation*} \\ 
\quad Since we get (1) in the last step, $(3^x -1 )\ mod(3^y -1) = 3^{(x \ mod \ y)} -1$ is true.\\


b) 
\begin{equation*}
\begin{split}
gcd(123,277) &= gcd(277,123 \ mod\ 277) \\
			 &= gcd(277,123) \\
			 &= gcd(123,277 \ mod\ 123) \\
			 &= gcd(123,31) \\
			 &= gcd(31,123 \ mod\ 31)  \\
			 &= gcd(31,30)  \\
			 &= gcd(30,31 \ mod\ 30)   \\
		 	 &= gcd(30,1)   \\
		 	 &= gcd(1,30 \ mod\ 1)    \\
			 &= gcd(1,0)    \\
			 &= 1           \\	
\end{split}	 
\end{equation*} \\ \\


\end{document}

​

