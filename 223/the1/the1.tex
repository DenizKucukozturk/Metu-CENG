\documentclass[12pt]{article}
\usepackage[utf8]{inputenc}
\usepackage{float}
\usepackage{amsmath}


\usepackage[hmargin=3cm,vmargin=6.0cm]{geometry}
%\topmargin=0cm
\topmargin=-2cm
\addtolength{\textheight}{6.5cm}
\addtolength{\textwidth}{2.0cm}
%\setlength{\leftmargin}{-5cm}
\setlength{\oddsidemargin}{0.0cm}
\setlength{\evensidemargin}{0.0cm}

%misc libraries goes here
%\usepackage{fitch}


\begin{document}

\section*{Student Information } 
%Write your full name and id number between the colon and newline
%Put one empty space character after colon and before newline
Full Name : Yavuz Selim YEŞİLYURT \\
Id Number : 2259166 \\

% Write your answers below the section tags
\section*{Answer 1}
\hspace{5mm} 1)\\
\begin{table}[H]
\small
\centering
\begin{tabular}{|c|c|c|c|c|c|c|}	
\hline
$p$ & $q$ & $\neg q$ & $p\rightarrow q$ & $\neg q\wedge(p \rightarrow q)$ & $\neg p$ & $ (\neg q\wedge(p \rightarrow q)) \rightarrow \neg p$\\
\hline 
T & T & F & T & F & F & T\\			
T & F & T & F & F & F & T\\
F & T & F & T & F & T & T\\
F & F & T & T & T & T & T\\
\hline 
\end{tabular}
\end{table}
2)\\
\begin{table}[H]
\small
\centering
\begin{tabular}{|c|c|c|c|c|c|c|c|c|}	
\hline
$p$ & $q$ & $r$ & $p \vee q$ & $\neg p$ & $\neg p \vee r$ & $(p \vee q) \wedge (\neg p \vee r) $ & $q \vee r$ & $((p \vee q) \wedge (\neg p \vee r)) \rightarrow q \vee r$\\
\hline 
T & T & T & T & F & T & T & T & T\\			
T & T & F & T & F & F & F & T & T\\			
T & F & F & T & F & F & F & F & T\\			
F & T & T & T & T & T & T & T & T\\		
F & T & F & T & T & T & T & T & T\\			
F & F & T & F & T & T & F & T & T\\			
T & F & T & T & F & T & T & T & T\\			
F & F & F & F & T & T & F & F &T\\		
\hline 
\end{tabular}
\end{table}


\section*{Answer 2}
\begin{equation*} 
\begin{split}
 (p \rightarrow q) \vee (p \rightarrow r) &\equiv (\neg p \vee q) \vee (p \rightarrow r)\ \qquad \quad \ table\ 7, Equivalence \ 1\\
 &\equiv (\neg p \vee q) \vee (\neg p \vee r)\ \quad \quad \quad table\ 7, Equivalence\ 1\\
 &\equiv (q \vee r) \vee (\neg p \vee \neg p) \quad \quad \quad \ table\ 6, Associative\ Law \\ &\equiv (q \vee r) \vee \neg p \ \ \ \quad \qquad \qquad \ table\ 6, Idempotent\ Law\\ &\equiv \neg(q \vee r) \rightarrow \neg p \ \ \ \quad \qquad \quad \ table\ 7, Equivalence\ 3\\ &\equiv (\neg q \wedge \neg r) \rightarrow \neg p \ \qquad \qquad \ table\ 6, De\ Morgan's\ Second\ Law\\
\end{split}
\end{equation*} 
\newpage
\section*{Answer 3}
\begin{enumerate}

	\item
	(a) All cats are friends with at least one dog.\\
	(b) Some cats are friends with all dogs.\\
	\item 
	(a) $\forall x \forall y ((Eats(x,y) z\wedge Meal(y)) \rightarrow Customer(x)) $ \\
	(b) $\exists x \exists y (Chef(x) \wedge Meal(y) \wedge \neg Cooks(x,y))$\\
	(c) $\exists x \forall y \exists z(((Cooks(x,y) \wedge Chef(x)) \rightarrow Meal(y)) \rightarrow (Eats(z,y) \wedge Customer(z)))$\\
	(d) $\forall x \exists y \exists z ((Chef(z) \wedge Chef(x) \wedge (x \neq z) \wedge Meal(y) \wedge \neg Cooks(x,y) \wedge Cooks(z,y)) \rightarrow Knows(x,z))$\\

\end{enumerate}

\section*{Answer 4}
$ \neg p$ and $ p \rightarrow q$ are given as premises and $ \neg q $ is given as conclusion. Since $ \neg p$ is given as true, the lefthandside of $p \rightarrow q$ is ($p$) false. For situations which $p$ is false (third and fourth rows on Table 1) $ p \rightarrow q$ returns true. But its truth value does not depend on $ q $. $q$ can be true or false. Therefore we can not deduce that $ \neg q$ is true. So this argument cannot be a deduction rule in a sound deductive system.\\
\begin{table}[H]
\small
\caption{ Truth table for $p \rightarrow q$ }
\centering
\begin{tabular}{|c|c|c|}	
\hline
$p$ & $q$ & $p\rightarrow q$\\
\hline 
T & T & T \\			
T & F & F \\
F & T & T \\
F & F & T \\
\hline 
\end{tabular}
\end{table}

\section*{Answer 5}
\begin{table}[H]
\begin{enumerate}
\item $ p \implies q \hfill premise$
\item $ q \implies r          \hfill    premise$
\item $ r \implies p      \hfill    premise$\\
\begin{tabular}{|p{10cm}|}
\hline
\item $ q                    \hfill    assumed$
\item $ r              \hfill        \implies e,2,4$
\item $ p                \hfill       \implies e,3,5$\\
\hline
\end{tabular}
\item $ q \implies p              \hfill     \implies i,4-6$
\item $ p \iff q        \hfill   \iff i,1,7$\\
\begin{tabular}{|p{10cm}|}
\hline
\item $ p       \hfill          assumed$
\item $ q          \hfill      \implies e,1,9$
\item $ r       \hfill  \implies e,2,10$\\
\hline
\end{tabular}
\item $ p \implies r        \hfill   \implies i,9-11$
\item $ p \iff r                 \hfill     \iff i,3,12$
\item $ (p \iff q) \land (p \iff r)          \hfill     \land i,8,13$
\end{enumerate}
\end{table}
%%%%%%%%%%%%%%%%%%%%%%
\section*{Answer 6}

\begin{table}[H]
\begin{enumerate}
\item $ \forall x (Q(x) \implies R(x)) \hfill  premise$
\item $ \exists x (P(x) \implies Q(x)) \hfill  premise$
\item $ \forall x (P(x)) \hfill   premise$\\
\begin{tabular}{|p{10cm}|}
\hline
\item $ P(c) \implies Q(c) \hfill   assumed $
\item $ P(c) \hfill   \forall e,3$
\item $ Q(c) \hfill   \implies e,4,5$
\item $ Q(c) \implies R(c) \hfill   \forall e,1$
\item $ R(c) \hfill   \implies e,6,7$
\item $ P(c) \land R(c) \hfill   \land i,5,8$
\item $ \exists x (P(x) \land R(x))  \hfill   \exists i,9$\\
\hline
\end{tabular}
\item $ \exists x (P(x) \land R(x)) \hfill   \exists e,2,4-10$
\end{enumerate}
\end{table}

\end{document}

​

​

