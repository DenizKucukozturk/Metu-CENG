\documentclass[12pt]{article}
\usepackage[utf8]{inputenc}
\usepackage{float}
\usepackage{amsmath}


\usepackage[hmargin=3cm,vmargin=6.0cm]{geometry}
%\topmargin=0cm
\topmargin=-2cm
\addtolength{\textheight}{6.5cm}
\addtolength{\textwidth}{2.0cm}
%\setlength{\leftmargin}{-5cm}
\setlength{\oddsidemargin}{0.0cm}
\setlength{\evensidemargin}{0.0cm}

\newcommand{\HRule}{\rule{\linewidth}{1mm}}

%misc libraries goes here
\usepackage{tikz}
\usetikzlibrary{automata,positioning}

\begin{document}

\noindent
\HRule \\[3mm]
\begin{flushright}

                                         \LARGE \textbf{CENG 222}  \\[4mm]
                                         \Large Statistical Methods for Computer Engineering \\[4mm]
                                        \normalsize      Spring '2018-2019 \\
                                           \Large   Homework 3 \\
\end{flushright}
\HRule

\section*{Student Information }
%Write your full name and id number between the colon and newline
%Put one empty space character after colon and before newline
Full Name : Yavuz Selim Yesilyurt \\
Id Number : 2259166 

% Write your answers below the section tags
\section*{Answer a}
I have conducted a Monte Carlo study using Matlab, after which I have used this study for estimating the probability that the total weight of all vehicles that pass over the bridge in the village in a day is more than 220 tons, for estimating expected weight and calculating the standard deviation of it. \\

To conduct such a study I have first used Normal approximation with $\alpha = 0.01$ and $\epsilon = 0.02$, namely (since no estimator for $p$ has been given I have directly used the following):

\begin{align*}
N &\geq 0.25(\frac{z_{\alpha/2}}{\epsilon})^2 \\
  &= 0.25(\frac{2.575}{0.02})^2 \\
  &\approx 4144
\end{align*}

I have created some variables for holding the values of distribution parameters and I have also created a vector named $TotalWeight$ for keeping the total weight of vehicles that use the bridge for each Monte Carlo run and initialized it to 0 for all $N$.\\

Next, to find number of vehicles for each type, I have generated samples ($NMotors$, $NCars$ and $NTrucks$) for all vehicles with their corresponding Poisson parameters using sampling from Poisson. \\

Then, to find weights of each vehicle according to its type, I have used the samples that correspond to numbers for each type of vehicles together with their corresponding Gamma parameters. With this way I was able to generate the sample weights for all vehicles ($WMotors$, $WCars$ and $WTrucks$) and after summing them up at the end I have calculated the total weight for 1 Monte Carlo run and filled the corresponding place in my $TotalWeight$ vector. I have repeated this study $N=4144$ times and filled the $TotalWeight$ vector accordingly. \\

For the answer of \textit{part a}; after construction of $TotalWeight$ vector with desired Monte Carlo runs, I have calculated the \textit{mean} of the proportion of runs with the total weight more than 220 tons. With this way I have estimated the probability that the total weight of all the vehicles that pass over the bridge in a day is more than 220 tons; in other words, I have found our estimator for the desired probability. \\

I have simulated my solution in Octave Online a number of times and I was able to determine that my estimated probability is always in between 0.35 and 0.38 (But in general 0.36). I share a sample output (which I will refer in other parts of the answer) in below:

\begin{center}
Estimated probability = 0.364865 \\
Expected weight = 208441.367130 \\
Standard deviation = 38401.600168 
\end{center}

\section*{Answer b}
For estimation of the total weight of all the vehicles that pass over the bridge in a day $X$, I have simply got the \textit{mean} of $TotalWeight$ and found the Expected weight. Expected weight for a sample simulation can be seen from the sample output shared in part a.

\section*{Answer c}
For estimation of $Std(X)$, I have simply got the \textit{std} of $TotalWeight$ and found the Standard deviation of $X$. Standard deviation for a sample simulation can be seen from the sample output shared in part a.\\

Since initially we have created a Monte Carlo study with size $N$ that attains our desired accuracy ($\alpha = 0.01$ and $\epsilon = 0.02$), We have guaranteed a Monte Carlo study of size $N$ with an error not exceeding $\epsilon$ with high probability $(1-\alpha)$ and created an estimator $X$ with that accuracy.

\end{document}
